\documentclass[11pt]{article}

    \usepackage[breakable]{tcolorbox}
    \usepackage{parskip} % Stop auto-indenting (to mimic markdown behaviour)
    \usepackage[utf8]{inputenc}
    \usepackage[portuguese]{babel}

    \usepackage{iftex}
    \ifPDFTeX
    	\usepackage[T1]{fontenc}
    	\usepackage{mathpazo}
    \else
    	\usepackage{fontspec}
    \fi

    % Basic figure setup, for now with no caption control since it's done
    % automatically by Pandoc (which extracts ![](path) syntax from Markdown).
    \usepackage{graphicx}
    % Maintain compatibility with old templates. Remove in nbconvert 6.0
    \let\Oldincludegraphics\includegraphics
    % Ensure that by default, figures have no caption (until we provide a
    % proper Figure object with a Caption API and a way to capture that
    % in the conversion process - todo).
    \usepackage{caption}
    \DeclareCaptionFormat{nocaption}{}
    \captionsetup{format=nocaption,aboveskip=0pt,belowskip=0pt}

    \usepackage[Export]{adjustbox} % Used to constrain images to a maximum size
    \adjustboxset{max size={0.9\linewidth}{0.9\paperheight}}
    \usepackage{float}
    \floatplacement{figure}{H} % forces figures to be placed at the correct location
    \usepackage{xcolor} % Allow colors to be defined
    \usepackage{enumerate} % Needed for markdown enumerations to work
    \usepackage{geometry} % Used to adjust the document margins
    \usepackage{amsmath} % Equations
    \usepackage{amssymb} % Equations
    \usepackage{textcomp} % defines textquotesingle
    % Hack from http://tex.stackexchange.com/a/47451/13684:
    \AtBeginDocument{%
        \def\PYZsq{\textquotesingle}% Upright quotes in Pygmentized code
    }
    \usepackage{upquote} % Upright quotes for verbatim code
    \usepackage{eurosym} % defines \euro
    \usepackage[mathletters]{ucs} % Extended unicode (utf-8) support
    \usepackage{fancyvrb} % verbatim replacement that allows latex
    \usepackage{grffile} % extends the file name processing of package graphics 
                         % to support a larger range
    \makeatletter % fix for grffile with XeLaTeX
    \def\Gread@@xetex#1{%
      \IfFileExists{"\Gin@base".bb}%
      {\Gread@eps{\Gin@base.bb}}%
      {\Gread@@xetex@aux#1}%
    }
    \makeatother

    % The hyperref package gives us a pdf with properly built
    % internal navigation ('pdf bookmarks' for the table of contents,
    % internal cross-reference links, web links for URLs, etc.)
    \usepackage{hyperref}
    % The default LaTeX title has an obnoxious amount of whitespace. By default,
    % titling removes some of it. It also provides customization options.
    \usepackage{titling}
    \usepackage{longtable} % longtable support required by pandoc >1.10
    \usepackage{booktabs}  % table support for pandoc > 1.12.2
    \usepackage[inline]{enumitem} % IRkernel/repr support (it uses the enumerate* environment)
    \usepackage[normalem]{ulem} % ulem is needed to support strikethroughs (\sout)
                                % normalem makes italics be italics, not underlines
    \usepackage{mathrsfs}
    

    
    % Colors for the hyperref package
    \definecolor{urlcolor}{rgb}{0,.145,.698}
    \definecolor{linkcolor}{rgb}{.71,0.21,0.01}
    \definecolor{citecolor}{rgb}{.12,.54,.11}

    % ANSI colors
    \definecolor{ansi-black}{HTML}{3E424D}
    \definecolor{ansi-black-intense}{HTML}{282C36}
    \definecolor{ansi-red}{HTML}{E75C58}
    \definecolor{ansi-red-intense}{HTML}{B22B31}
    \definecolor{ansi-green}{HTML}{00A250}
    \definecolor{ansi-green-intense}{HTML}{007427}
    \definecolor{ansi-yellow}{HTML}{DDB62B}
    \definecolor{ansi-yellow-intense}{HTML}{B27D12}
    \definecolor{ansi-blue}{HTML}{208FFB}
    \definecolor{ansi-blue-intense}{HTML}{0065CA}
    \definecolor{ansi-magenta}{HTML}{D160C4}
    \definecolor{ansi-magenta-intense}{HTML}{A03196}
    \definecolor{ansi-cyan}{HTML}{60C6C8}
    \definecolor{ansi-cyan-intense}{HTML}{258F8F}
    \definecolor{ansi-white}{HTML}{C5C1B4}
    \definecolor{ansi-white-intense}{HTML}{A1A6B2}
    \definecolor{ansi-default-inverse-fg}{HTML}{FFFFFF}
    \definecolor{ansi-default-inverse-bg}{HTML}{000000}

    % commands and environments needed by pandoc snippets
    % extracted from the output of `pandoc -s`
    \providecommand{\tightlist}{%
      \setlength{\itemsep}{0pt}\setlength{\parskip}{0pt}}
    \DefineVerbatimEnvironment{Highlighting}{Verbatim}{commandchars=\\\{\}}
    % Add ',fontsize=\small' for more characters per line
    \newenvironment{Shaded}{}{}
    \newcommand{\KeywordTok}[1]{\textcolor[rgb]{0.00,0.44,0.13}{\textbf{{#1}}}}
    \newcommand{\DataTypeTok}[1]{\textcolor[rgb]{0.56,0.13,0.00}{{#1}}}
    \newcommand{\DecValTok}[1]{\textcolor[rgb]{0.25,0.63,0.44}{{#1}}}
    \newcommand{\BaseNTok}[1]{\textcolor[rgb]{0.25,0.63,0.44}{{#1}}}
    \newcommand{\FloatTok}[1]{\textcolor[rgb]{0.25,0.63,0.44}{{#1}}}
    \newcommand{\CharTok}[1]{\textcolor[rgb]{0.25,0.44,0.63}{{#1}}}
    \newcommand{\StringTok}[1]{\textcolor[rgb]{0.25,0.44,0.63}{{#1}}}
    \newcommand{\CommentTok}[1]{\textcolor[rgb]{0.38,0.63,0.69}{\textit{{#1}}}}
    \newcommand{\OtherTok}[1]{\textcolor[rgb]{0.00,0.44,0.13}{{#1}}}
    \newcommand{\AlertTok}[1]{\textcolor[rgb]{1.00,0.00,0.00}{\textbf{{#1}}}}
    \newcommand{\FunctionTok}[1]{\textcolor[rgb]{0.02,0.16,0.49}{{#1}}}
    \newcommand{\RegionMarkerTok}[1]{{#1}}
    \newcommand{\ErrorTok}[1]{\textcolor[rgb]{1.00,0.00,0.00}{\textbf{{#1}}}}
    \newcommand{\NormalTok}[1]{{#1}}
    
    % Additional commands for more recent versions of Pandoc
    \newcommand{\ConstantTok}[1]{\textcolor[rgb]{0.53,0.00,0.00}{{#1}}}
    \newcommand{\SpecialCharTok}[1]{\textcolor[rgb]{0.25,0.44,0.63}{{#1}}}
    \newcommand{\VerbatimStringTok}[1]{\textcolor[rgb]{0.25,0.44,0.63}{{#1}}}
    \newcommand{\SpecialStringTok}[1]{\textcolor[rgb]{0.73,0.40,0.53}{{#1}}}
    \newcommand{\ImportTok}[1]{{#1}}
    \newcommand{\DocumentationTok}[1]{\textcolor[rgb]{0.73,0.13,0.13}{\textit{{#1}}}}
    \newcommand{\AnnotationTok}[1]{\textcolor[rgb]{0.38,0.63,0.69}{\textbf{\textit{{#1}}}}}
    \newcommand{\CommentVarTok}[1]{\textcolor[rgb]{0.38,0.63,0.69}{\textbf{\textit{{#1}}}}}
    \newcommand{\VariableTok}[1]{\textcolor[rgb]{0.10,0.09,0.49}{{#1}}}
    \newcommand{\ControlFlowTok}[1]{\textcolor[rgb]{0.00,0.44,0.13}{\textbf{{#1}}}}
    \newcommand{\OperatorTok}[1]{\textcolor[rgb]{0.40,0.40,0.40}{{#1}}}
    \newcommand{\BuiltInTok}[1]{{#1}}
    \newcommand{\ExtensionTok}[1]{{#1}}
    \newcommand{\PreprocessorTok}[1]{\textcolor[rgb]{0.74,0.48,0.00}{{#1}}}
    \newcommand{\AttributeTok}[1]{\textcolor[rgb]{0.49,0.56,0.16}{{#1}}}
    \newcommand{\InformationTok}[1]{\textcolor[rgb]{0.38,0.63,0.69}{\textbf{\textit{{#1}}}}}
    \newcommand{\WarningTok}[1]{\textcolor[rgb]{0.38,0.63,0.69}{\textbf{\textit{{#1}}}}}
    
    
    % Define a nice break command that doesn't care if a line doesn't already
    % exist.
    \def\br{\hspace*{\fill} \\* }
    % Math Jax compatibility definitions
    \def\gt{>}
    \def\lt{<}
    \let\Oldtex\TeX
    \let\Oldlatex\LaTeX
    \renewcommand{\TeX}{\textrm{\Oldtex}}
    \renewcommand{\LaTeX}{\textrm{\Oldlatex}}
    % Document parameters
    % Document title
    \title{Otimização-não-linear-final-Ericles-9790687}
    
    
    
    
    
% Pygments definitions
\makeatletter
\def\PY@reset{\let\PY@it=\relax \let\PY@bf=\relax%
    \let\PY@ul=\relax \let\PY@tc=\relax%
    \let\PY@bc=\relax \let\PY@ff=\relax}
\def\PY@tok#1{\csname PY@tok@#1\endcsname}
\def\PY@toks#1+{\ifx\relax#1\empty\else%
    \PY@tok{#1}\expandafter\PY@toks\fi}
\def\PY@do#1{\PY@bc{\PY@tc{\PY@ul{%
    \PY@it{\PY@bf{\PY@ff{#1}}}}}}}
\def\PY#1#2{\PY@reset\PY@toks#1+\relax+\PY@do{#2}}

\expandafter\def\csname PY@tok@w\endcsname{\def\PY@tc##1{\textcolor[rgb]{0.73,0.73,0.73}{##1}}}
\expandafter\def\csname PY@tok@c\endcsname{\let\PY@it=\textit\def\PY@tc##1{\textcolor[rgb]{0.25,0.50,0.50}{##1}}}
\expandafter\def\csname PY@tok@cp\endcsname{\def\PY@tc##1{\textcolor[rgb]{0.74,0.48,0.00}{##1}}}
\expandafter\def\csname PY@tok@k\endcsname{\let\PY@bf=\textbf\def\PY@tc##1{\textcolor[rgb]{0.00,0.50,0.00}{##1}}}
\expandafter\def\csname PY@tok@kp\endcsname{\def\PY@tc##1{\textcolor[rgb]{0.00,0.50,0.00}{##1}}}
\expandafter\def\csname PY@tok@kt\endcsname{\def\PY@tc##1{\textcolor[rgb]{0.69,0.00,0.25}{##1}}}
\expandafter\def\csname PY@tok@o\endcsname{\def\PY@tc##1{\textcolor[rgb]{0.40,0.40,0.40}{##1}}}
\expandafter\def\csname PY@tok@ow\endcsname{\let\PY@bf=\textbf\def\PY@tc##1{\textcolor[rgb]{0.67,0.13,1.00}{##1}}}
\expandafter\def\csname PY@tok@nb\endcsname{\def\PY@tc##1{\textcolor[rgb]{0.00,0.50,0.00}{##1}}}
\expandafter\def\csname PY@tok@nf\endcsname{\def\PY@tc##1{\textcolor[rgb]{0.00,0.00,1.00}{##1}}}
\expandafter\def\csname PY@tok@nc\endcsname{\let\PY@bf=\textbf\def\PY@tc##1{\textcolor[rgb]{0.00,0.00,1.00}{##1}}}
\expandafter\def\csname PY@tok@nn\endcsname{\let\PY@bf=\textbf\def\PY@tc##1{\textcolor[rgb]{0.00,0.00,1.00}{##1}}}
\expandafter\def\csname PY@tok@ne\endcsname{\let\PY@bf=\textbf\def\PY@tc##1{\textcolor[rgb]{0.82,0.25,0.23}{##1}}}
\expandafter\def\csname PY@tok@nv\endcsname{\def\PY@tc##1{\textcolor[rgb]{0.10,0.09,0.49}{##1}}}
\expandafter\def\csname PY@tok@no\endcsname{\def\PY@tc##1{\textcolor[rgb]{0.53,0.00,0.00}{##1}}}
\expandafter\def\csname PY@tok@nl\endcsname{\def\PY@tc##1{\textcolor[rgb]{0.63,0.63,0.00}{##1}}}
\expandafter\def\csname PY@tok@ni\endcsname{\let\PY@bf=\textbf\def\PY@tc##1{\textcolor[rgb]{0.60,0.60,0.60}{##1}}}
\expandafter\def\csname PY@tok@na\endcsname{\def\PY@tc##1{\textcolor[rgb]{0.49,0.56,0.16}{##1}}}
\expandafter\def\csname PY@tok@nt\endcsname{\let\PY@bf=\textbf\def\PY@tc##1{\textcolor[rgb]{0.00,0.50,0.00}{##1}}}
\expandafter\def\csname PY@tok@nd\endcsname{\def\PY@tc##1{\textcolor[rgb]{0.67,0.13,1.00}{##1}}}
\expandafter\def\csname PY@tok@s\endcsname{\def\PY@tc##1{\textcolor[rgb]{0.73,0.13,0.13}{##1}}}
\expandafter\def\csname PY@tok@sd\endcsname{\let\PY@it=\textit\def\PY@tc##1{\textcolor[rgb]{0.73,0.13,0.13}{##1}}}
\expandafter\def\csname PY@tok@si\endcsname{\let\PY@bf=\textbf\def\PY@tc##1{\textcolor[rgb]{0.73,0.40,0.53}{##1}}}
\expandafter\def\csname PY@tok@se\endcsname{\let\PY@bf=\textbf\def\PY@tc##1{\textcolor[rgb]{0.73,0.40,0.13}{##1}}}
\expandafter\def\csname PY@tok@sr\endcsname{\def\PY@tc##1{\textcolor[rgb]{0.73,0.40,0.53}{##1}}}
\expandafter\def\csname PY@tok@ss\endcsname{\def\PY@tc##1{\textcolor[rgb]{0.10,0.09,0.49}{##1}}}
\expandafter\def\csname PY@tok@sx\endcsname{\def\PY@tc##1{\textcolor[rgb]{0.00,0.50,0.00}{##1}}}
\expandafter\def\csname PY@tok@m\endcsname{\def\PY@tc##1{\textcolor[rgb]{0.40,0.40,0.40}{##1}}}
\expandafter\def\csname PY@tok@gh\endcsname{\let\PY@bf=\textbf\def\PY@tc##1{\textcolor[rgb]{0.00,0.00,0.50}{##1}}}
\expandafter\def\csname PY@tok@gu\endcsname{\let\PY@bf=\textbf\def\PY@tc##1{\textcolor[rgb]{0.50,0.00,0.50}{##1}}}
\expandafter\def\csname PY@tok@gd\endcsname{\def\PY@tc##1{\textcolor[rgb]{0.63,0.00,0.00}{##1}}}
\expandafter\def\csname PY@tok@gi\endcsname{\def\PY@tc##1{\textcolor[rgb]{0.00,0.63,0.00}{##1}}}
\expandafter\def\csname PY@tok@gr\endcsname{\def\PY@tc##1{\textcolor[rgb]{1.00,0.00,0.00}{##1}}}
\expandafter\def\csname PY@tok@ge\endcsname{\let\PY@it=\textit}
\expandafter\def\csname PY@tok@gs\endcsname{\let\PY@bf=\textbf}
\expandafter\def\csname PY@tok@gp\endcsname{\let\PY@bf=\textbf\def\PY@tc##1{\textcolor[rgb]{0.00,0.00,0.50}{##1}}}
\expandafter\def\csname PY@tok@go\endcsname{\def\PY@tc##1{\textcolor[rgb]{0.53,0.53,0.53}{##1}}}
\expandafter\def\csname PY@tok@gt\endcsname{\def\PY@tc##1{\textcolor[rgb]{0.00,0.27,0.87}{##1}}}
\expandafter\def\csname PY@tok@err\endcsname{\def\PY@bc##1{\setlength{\fboxsep}{0pt}\fcolorbox[rgb]{1.00,0.00,0.00}{1,1,1}{\strut ##1}}}
\expandafter\def\csname PY@tok@kc\endcsname{\let\PY@bf=\textbf\def\PY@tc##1{\textcolor[rgb]{0.00,0.50,0.00}{##1}}}
\expandafter\def\csname PY@tok@kd\endcsname{\let\PY@bf=\textbf\def\PY@tc##1{\textcolor[rgb]{0.00,0.50,0.00}{##1}}}
\expandafter\def\csname PY@tok@kn\endcsname{\let\PY@bf=\textbf\def\PY@tc##1{\textcolor[rgb]{0.00,0.50,0.00}{##1}}}
\expandafter\def\csname PY@tok@kr\endcsname{\let\PY@bf=\textbf\def\PY@tc##1{\textcolor[rgb]{0.00,0.50,0.00}{##1}}}
\expandafter\def\csname PY@tok@bp\endcsname{\def\PY@tc##1{\textcolor[rgb]{0.00,0.50,0.00}{##1}}}
\expandafter\def\csname PY@tok@fm\endcsname{\def\PY@tc##1{\textcolor[rgb]{0.00,0.00,1.00}{##1}}}
\expandafter\def\csname PY@tok@vc\endcsname{\def\PY@tc##1{\textcolor[rgb]{0.10,0.09,0.49}{##1}}}
\expandafter\def\csname PY@tok@vg\endcsname{\def\PY@tc##1{\textcolor[rgb]{0.10,0.09,0.49}{##1}}}
\expandafter\def\csname PY@tok@vi\endcsname{\def\PY@tc##1{\textcolor[rgb]{0.10,0.09,0.49}{##1}}}
\expandafter\def\csname PY@tok@vm\endcsname{\def\PY@tc##1{\textcolor[rgb]{0.10,0.09,0.49}{##1}}}
\expandafter\def\csname PY@tok@sa\endcsname{\def\PY@tc##1{\textcolor[rgb]{0.73,0.13,0.13}{##1}}}
\expandafter\def\csname PY@tok@sb\endcsname{\def\PY@tc##1{\textcolor[rgb]{0.73,0.13,0.13}{##1}}}
\expandafter\def\csname PY@tok@sc\endcsname{\def\PY@tc##1{\textcolor[rgb]{0.73,0.13,0.13}{##1}}}
\expandafter\def\csname PY@tok@dl\endcsname{\def\PY@tc##1{\textcolor[rgb]{0.73,0.13,0.13}{##1}}}
\expandafter\def\csname PY@tok@s2\endcsname{\def\PY@tc##1{\textcolor[rgb]{0.73,0.13,0.13}{##1}}}
\expandafter\def\csname PY@tok@sh\endcsname{\def\PY@tc##1{\textcolor[rgb]{0.73,0.13,0.13}{##1}}}
\expandafter\def\csname PY@tok@s1\endcsname{\def\PY@tc##1{\textcolor[rgb]{0.73,0.13,0.13}{##1}}}
\expandafter\def\csname PY@tok@mb\endcsname{\def\PY@tc##1{\textcolor[rgb]{0.40,0.40,0.40}{##1}}}
\expandafter\def\csname PY@tok@mf\endcsname{\def\PY@tc##1{\textcolor[rgb]{0.40,0.40,0.40}{##1}}}
\expandafter\def\csname PY@tok@mh\endcsname{\def\PY@tc##1{\textcolor[rgb]{0.40,0.40,0.40}{##1}}}
\expandafter\def\csname PY@tok@mi\endcsname{\def\PY@tc##1{\textcolor[rgb]{0.40,0.40,0.40}{##1}}}
\expandafter\def\csname PY@tok@il\endcsname{\def\PY@tc##1{\textcolor[rgb]{0.40,0.40,0.40}{##1}}}
\expandafter\def\csname PY@tok@mo\endcsname{\def\PY@tc##1{\textcolor[rgb]{0.40,0.40,0.40}{##1}}}
\expandafter\def\csname PY@tok@ch\endcsname{\let\PY@it=\textit\def\PY@tc##1{\textcolor[rgb]{0.25,0.50,0.50}{##1}}}
\expandafter\def\csname PY@tok@cm\endcsname{\let\PY@it=\textit\def\PY@tc##1{\textcolor[rgb]{0.25,0.50,0.50}{##1}}}
\expandafter\def\csname PY@tok@cpf\endcsname{\let\PY@it=\textit\def\PY@tc##1{\textcolor[rgb]{0.25,0.50,0.50}{##1}}}
\expandafter\def\csname PY@tok@c1\endcsname{\let\PY@it=\textit\def\PY@tc##1{\textcolor[rgb]{0.25,0.50,0.50}{##1}}}
\expandafter\def\csname PY@tok@cs\endcsname{\let\PY@it=\textit\def\PY@tc##1{\textcolor[rgb]{0.25,0.50,0.50}{##1}}}

\def\PYZbs{\char`\\}
\def\PYZus{\char`\_}
\def\PYZob{\char`\{}
\def\PYZcb{\char`\}}
\def\PYZca{\char`\^}
\def\PYZam{\char`\&}
\def\PYZlt{\char`\<}
\def\PYZgt{\char`\>}
\def\PYZsh{\char`\#}
\def\PYZpc{\char`\%}
\def\PYZdl{\char`\$}
\def\PYZhy{\char`\-}
\def\PYZsq{\char`\'}
\def\PYZdq{\char`\"}
\def\PYZti{\char`\~}
% for compatibility with earlier versions
\def\PYZat{@}
\def\PYZlb{[}
\def\PYZrb{]}
\makeatother


    % For linebreaks inside Verbatim environment from package fancyvrb. 
    \makeatletter
        \newbox\Wrappedcontinuationbox 
        \newbox\Wrappedvisiblespacebox 
        \newcommand*\Wrappedvisiblespace {\textcolor{red}{\textvisiblespace}} 
        \newcommand*\Wrappedcontinuationsymbol {\textcolor{red}{\llap{\tiny$\m@th\hookrightarrow$}}} 
        \newcommand*\Wrappedcontinuationindent {3ex } 
        \newcommand*\Wrappedafterbreak {\kern\Wrappedcontinuationindent\copy\Wrappedcontinuationbox} 
        % Take advantage of the already applied Pygments mark-up to insert 
        % potential linebreaks for TeX processing. 
        %        {, <, #, %, $, ' and ": go to next line. 
        %        _, }, ^, &, >, - and ~: stay at end of broken line. 
        % Use of \textquotesingle for straight quote. 
        \newcommand*\Wrappedbreaksatspecials {% 
            \def\PYGZus{\discretionary{\char`\_}{\Wrappedafterbreak}{\char`\_}}% 
            \def\PYGZob{\discretionary{}{\Wrappedafterbreak\char`\{}{\char`\{}}% 
            \def\PYGZcb{\discretionary{\char`\}}{\Wrappedafterbreak}{\char`\}}}% 
            \def\PYGZca{\discretionary{\char`\^}{\Wrappedafterbreak}{\char`\^}}% 
            \def\PYGZam{\discretionary{\char`\&}{\Wrappedafterbreak}{\char`\&}}% 
            \def\PYGZlt{\discretionary{}{\Wrappedafterbreak\char`\<}{\char`\<}}% 
            \def\PYGZgt{\discretionary{\char`\>}{\Wrappedafterbreak}{\char`\>}}% 
            \def\PYGZsh{\discretionary{}{\Wrappedafterbreak\char`\#}{\char`\#}}% 
            \def\PYGZpc{\discretionary{}{\Wrappedafterbreak\char`\%}{\char`\%}}% 
            \def\PYGZdl{\discretionary{}{\Wrappedafterbreak\char`\$}{\char`\$}}% 
            \def\PYGZhy{\discretionary{\char`\-}{\Wrappedafterbreak}{\char`\-}}% 
            \def\PYGZsq{\discretionary{}{\Wrappedafterbreak\textquotesingle}{\textquotesingle}}% 
            \def\PYGZdq{\discretionary{}{\Wrappedafterbreak\char`\"}{\char`\"}}% 
            \def\PYGZti{\discretionary{\char`\~}{\Wrappedafterbreak}{\char`\~}}% 
        } 
        % Some characters . , ; ? ! / are not pygmentized. 
        % This macro makes them "active" and they will insert potential linebreaks 
        \newcommand*\Wrappedbreaksatpunct {% 
            \lccode`\~`\.\lowercase{\def~}{\discretionary{\hbox{\char`\.}}{\Wrappedafterbreak}{\hbox{\char`\.}}}% 
            \lccode`\~`\,\lowercase{\def~}{\discretionary{\hbox{\char`\,}}{\Wrappedafterbreak}{\hbox{\char`\,}}}% 
            \lccode`\~`\;\lowercase{\def~}{\discretionary{\hbox{\char`\;}}{\Wrappedafterbreak}{\hbox{\char`\;}}}% 
            \lccode`\~`\:\lowercase{\def~}{\discretionary{\hbox{\char`\:}}{\Wrappedafterbreak}{\hbox{\char`\:}}}% 
            \lccode`\~`\?\lowercase{\def~}{\discretionary{\hbox{\char`\?}}{\Wrappedafterbreak}{\hbox{\char`\?}}}% 
            \lccode`\~`\!\lowercase{\def~}{\discretionary{\hbox{\char`\!}}{\Wrappedafterbreak}{\hbox{\char`\!}}}% 
            \lccode`\~`\/\lowercase{\def~}{\discretionary{\hbox{\char`\/}}{\Wrappedafterbreak}{\hbox{\char`\/}}}% 
            \catcode`\.\active
            \catcode`\,\active 
            \catcode`\;\active
            \catcode`\:\active
            \catcode`\?\active
            \catcode`\!\active
            \catcode`\/\active 
            \lccode`\~`\~ 	
        }
    \makeatother

    \let\OriginalVerbatim=\Verbatim
    \makeatletter
    \renewcommand{\Verbatim}[1][1]{%
        %\parskip\z@skip
        \sbox\Wrappedcontinuationbox {\Wrappedcontinuationsymbol}%
        \sbox\Wrappedvisiblespacebox {\FV@SetupFont\Wrappedvisiblespace}%
        \def\FancyVerbFormatLine ##1{\hsize\linewidth
            \vtop{\raggedright\hyphenpenalty\z@\exhyphenpenalty\z@
                \doublehyphendemerits\z@\finalhyphendemerits\z@
                \strut ##1\strut}%
        }%
        % If the linebreak is at a space, the latter will be displayed as visible
        % space at end of first line, and a continuation symbol starts next line.
        % Stretch/shrink are however usually zero for typewriter font.
        \def\FV@Space {%
            \nobreak\hskip\z@ plus\fontdimen3\font minus\fontdimen4\font
            \discretionary{\copy\Wrappedvisiblespacebox}{\Wrappedafterbreak}
            {\kern\fontdimen2\font}%
        }%
        
        % Allow breaks at special characters using \PYG... macros.
        \Wrappedbreaksatspecials
        % Breaks at punctuation characters . , ; ? ! and / need catcode=\active 	
        \OriginalVerbatim[#1,codes*=\Wrappedbreaksatpunct]%
    }
    \makeatother

    % Exact colors from NB
    \definecolor{incolor}{HTML}{303F9F}
    \definecolor{outcolor}{HTML}{D84315}
    \definecolor{cellborder}{HTML}{CFCFCF}
    \definecolor{cellbackground}{HTML}{F7F7F7}
    
    % prompt
    \makeatletter
    \newcommand{\boxspacing}{\kern\kvtcb@left@rule\kern\kvtcb@boxsep}
    \makeatother
    \newcommand{\prompt}[4]{
        \ttfamily\llap{{\color{#2}[#3]:\hspace{3pt}#4}}\vspace{-\baselineskip}
    }
    

    
    % Prevent overflowing lines due to hard-to-break entities
    \sloppy 
    % Setup hyperref package
    \hypersetup{
      breaklinks=true,  % so long urls are correctly broken across lines
      colorlinks=true,
      urlcolor=urlcolor,
      linkcolor=linkcolor,
      citecolor=citecolor,
      }
    % Slightly bigger margins than the latex defaults
    
    \geometry{verbose,tmargin=1in,bmargin=1in,lmargin=1in,rmargin=1in}
    
    

\begin{document}
    
    \maketitle
    
    

    
    \section{Método dos gradientes
conjugados}\label{muxe9todo-dos-gradientes-conjugados}

    \subsection{Introdução}\label{introduuxe7uxe3o}

    O método dos gradientes conjugados (daqui para frente abreviado como CG,
da sigla em inglês) é o método iterativo mais popular para resolver
sistemas grandes de equações lineares. Esse método é particularmente
efetivo para resolver sistemas \[Ax = b\] onde \(x\) é um vetor
desconhecido (a incógnita), \(b\) é conhecido e \(A\) é uma matriz
positiva-definida. Esse método é mais adequado para quando \(A\) for
esparsa: quando \(A\) é densa, provavelmente um método mais adequado
seria alguma fatoração (como a \(LU\)); mas quando \(A\) é esparsa, suas
fatorações geralmente são densas, impondo um custo de memória que pode
ser inaceitável.

    Apesar de CG ser usado para resolver sistemas de equações lineares, um
melhor entendimento sobre o método vem ao considerarmos um problema de
otimização quadrática \[
min \quad f(x) = \frac12(x,x)_A - (b,x) + c
\]

onde \((x,y)_A := (x,Ay)\) e \((x,y)\) é o produto interno usual. Temos
que a derivada de \(f\) em \(x\), \(D(f)(x)\), é
\(\frac12 (A'+ A)x - b\), que, quando \(A\) é simétrica, é \(Ax - b\). O
mínimo (ou máximo) \(x^*\) da função é obtido onde sua derivada se
anula, do que segue que resolvemos para\(Ax^* - b = 0\), rendendo o
sistema linear inicial. Se \(A\), além de simétrica, é
positiva-definida, \(x^*\) é um mínimo. Se \(A\) não for simétrica, CG
vai buscar solução para \(\frac12(A' + A)x = b\) (note que \(A' + A\) é
simétrica).

Se \(A\) é positiva-definida, sua forma bilinear correspondente
\((x,A'x)\) (ou melhor, o seu gráfico) corresponde a um parabolóide, do
que é intuitivo que \(x^*\) é único.

    \subsection{Ideia geral do método}\label{ideia-geral-do-muxe9todo}

    A ideia básica desse método (assim como a de outros métodos) é

\begin{enumerate}
\def\labelenumi{\arabic{enumi}.}
\tightlist
\item
  Começando de um \(x_0\);
\item
  Escolha uma direção \(d_i\);
\item
  Escolha quanto se quer "andar" na direção \(d_i\), determinando o
  tamanho de passo \(\alpha\);
\item
  Faça \(x_i := x_{i-1} + \alpha d_i\).
\end{enumerate}

Assim, podemos definir os vetores

\begin{itemize}
\tightlist
\item
  Resíduo: \(r_i := b - Ax_i\);
\item
  Erro: \(e_i := x_i - x\) (note que \(r_i = Ae_i\))
\end{itemize}

Note que \(r_i = - D(f)(x_i)\), que é a direção em \(x_i\) que aponta
onde a função está diminuindo mais rapidamente - que, o que é
importante, é ortogonal à curva de nível em \(x_i\).

O método de máxima descida, no qual CG se baseia, faz \(d_i = r_{i-1}\),
e busca um passo \(\alpha\) que minimize a função nessa direção. Deve
ser intuitivo que essa minimização ocorre quando \(\alpha\) for tal que
faça \(r_{i-1} \perp r_i\) (de modo que \(x_{i-1} + \alpha d_i\)
tangencie uma curva de nível de \(f\)). Uma conta rápida mostra que
\(\alpha = \frac{(r_{i-1},r_{i-1})}{(r_{i-1},r_{i-1})_A}\).

    O método de máxima descida no geral leva mais de uma iteração para
encontrar a solução. Na verdade, máxima descida converge em apenas uma
iteração apenas se o chute inicial \(x_0\) estiver na direção de um dos
auto-vetores de \(A\), que coincide com um dos eixos do elipsóide (o que
raramente ocorre, a não ser que \(A\) seja múltipla da identidade,
fazendo com que as curvas de nível de sua forma bilinear sejam esferas,
e seus auto-valores sejam todos iguais).

    

    \subsubsection{Gradientes conjugados}\label{gradientes-conjugados}

A ideia do método de gradientes conjugados é buscar pela solução
iterativamente por \(n\) direções de busca \(\{d_0, \ldots, d_n\}\), de
modo que se possa garantir que \(x_i\) seja a melhor solução para o
problema em \(\mathcal D_i = span(\{d_0, \ldots, d_i\})\).

\subsubsection{Conjugação de
Gram-Schmidt}\label{conjugauxe7uxe3o-de-gram-schmidt}

Em particular, fazemos as direções de busca \(d_i\) A-ortogonais (ou
\emph{conjugados}) entre si, ie. \((d_i,d_j)_A = 0\) se \(i \neq j\).
Podemos pensar na A-ortogonalidade como uma ortogonalidade usual sob uma
mudança de coordenadas que transforma os elipsóides das curvas de nível
em esferas, que é do que se deriva boa parte das propriedades que
obtemos de CG.

Para obter essas direções, podemos prosseguir pelo processo de
Gram-Schmidt, mas tomando o produto interno como sendo o A-produto
interno \((\cdot,\cdot)_A\), descrito brevemente a seguir:

Tome um conjunto de \(n\) vetores linearmente independentes \(\{u_i\}\).
Para obter \(d_i\) bastar tomar os \(u_i\), em ordem, e subtrair os
componentes que não sejam A-ortogonais com os \(d_i\) anteriores. Esse é
um algoritmo incremental:

\[
\begin{split}
    d_0 &= u_0 \\
    \text{para i > 0} \quad d_i &= u_i - \sum_{k=0}^{i-1}\beta_{ik}d_k \\
    \beta_{ij} &= \frac{(u_i, d_j)_A}{(d_j,d_j)_A}
\end{split}
\]

    Em CG, faz-se \(u_i = r_i\). Como \(r_{i+1} = r_i - \alpha_i Ad_i\),
temos que cada \(r_i\) é uma combinação linear do resíduo anterior e
\(Ad_{i-1}\). Disso, obtemos então que \(\mathcal D_i\), o "espaço de
busca" até o passo i, é formado pela união de \(\mathcal D_i\) e
\(A \mathcal D_i\), do que segue que
\[\mathcal D_i = span\{d_0, Ad_0, A^2d_0, \ldots, A^{i-1}d_0\} = span\{r_0, Ar_0, A^2r_0, \ldots, A^{i-1}r_0\}. \]
Esses espaços são chamados espaços de Krylov (mais em geral,
\(Krylov_r(A,b) := \{A^jb|j<r\}\)).

Disso tiramos a propriedade de que, como \(r_{i+1}\) é ortogonal a
\(\mathcal D_{i+1}\) (o que segue do fato de o resíduo em qualquer ponto
ser ortogonal á superfície elipsoidal naquele ponto, e de o hiperplano
\(x_0 + \mathcal D_i\) também o ser), \(r_{i+1}\) é A-ortogonal a
\(\mathcal D_i \subset D_{i+1}\). Assim, \(r_{i+1}\) é A-ortogonal a
todas as direções de busca anteriores, exceto \(d_i\), o que facilita
grandemente o processo de Gram-Schmidt descrito acima.

Mais especificamente, obtemos que \[
(r_i, d_j)_A = \begin{cases} \frac1{\alpha_i} (r_i,r_i), & i = j \\
                            -\frac1{\alpha_{i-1}} (r_i,r_i), & i = j + 1\\ 
                              0, & \text{caso contrário}
                              \end{cases}
\]

o que simplifica grandemente a expressão de \(\beta_{ij}\), levando a
complexidade (tanto em tempo quanto em espaço) de CG de \(O(n^2)\) para
\(O(m)\), onde m é a quantidade de elementos não nulos de \(A\). Mais
especificamente, obtemos \[
\beta_i := \beta_{i,i-1} = \frac{(r_i,r_i)}{(r_{i-1},r_{i-1})}
\]

    \subsubsection{Precondicionador}\label{precondicionador}

    Dada uma matriz \(A\) cujos maior e menor autovalores são,
respectivamente, \(\lambda_1\) e \(\lambda_n\), definimos seu
\emph{número de condição}, ou simplesmente seu \emph{condicionante} como
\(\kappa(A) := \frac{\lambda_1}{\lambda_n}\). O condicionante de uma
matriz tem um papel importante na velocidade de convergência de diversos
métodos e, em particular, para CG. Em geral valores grandes para
\(\kappa\) são ruins, fazendo com que o método possa levar a uma
convergência inaceitavelmente lenta.

    Para lidar com esse problema, uma alternativa é a pre-multiplicação do
sistema por uma matriz \(M\) positiva-definida que aproxima \(A\), e que
seja de fácil inversão. Então, pode-se resolver \(Ax = b\) de forma
indireta, fazendo \[
M^{-1}Ax = M^{-1}b
\] Se \(\kappa(M^{-1}A) \ll \kappa(A)\), o problema acima se torna muito
mais fácil do que o problema original.

O uso de precondicionadores apresenta, contudo, alguns problemas de
ordem prática e teórica. Um problema é que, mesmo \(A\) e \(M\) sendo
simétricas e definidas, \(M^{-1}A\) pode não ser. Isso pode ser
contornado, usando a decomposição de Cholesky, obtendo \(E\) tal que
\(EE' = M\), e \(E^{-1}AE^{-1}'\) será uma simétrica e
positiva-definida.

Qual precondicionador usar é um problema de ordem prática, existindo
algumas opções comuns, a depender do problema em mãos. Uma opção
simples, mas de resultados mediocres, é o precondicionador de Jacobi, em
que \(M\) é uma matriz diagonal (realmente, muito fácil de inverter).
Outra opção é precondicionamento de Cholesky incompleto, em que é feita
uma decomposição de Cholesky incompleta da matriz A

Mas vale dizer que é, em geral, entendido que CG é sempre feito com o
uso de algum precondicionador, ao menos para problemas de grandes, o que
motiva a não-omissão desse tema neste documento introdutório.

    \subsection{Algoritmo}\label{algoritmo}

O método de gradientes conjugados pode então ser resumido no seguinte
algoritmo (em que foram feitas algumas manipulações algébricas para
comportar o condicionador \(M\), sem a necessidade de computar \(E\)).

Dados \(A\), \(b\), um valor inicial para \(x\) uma quantidade máxima de
iterações \(i_{max}\) e tolerância de erro \(\epsilon < 1\) e um
precondicionador \(M\),

\[
\begin{split}
 &i \leftarrow 0 \\
 &r \leftarrow b - Ax \\
 &d \leftarrow M^{-1}r \\
 &\delta_{novo} \leftarrow (r,d) \\
 &\delta_0 \leftarrow \delta_{novo} \\
 & \text{Enquanto}\quad i < i_{max}\quad e \quad\delta_{novo} > \epsilon^2\delta_0 \quad\text{faça }\\
 & \qquad q \leftarrow Ad \\
 & \qquad \alpha \leftarrow \frac{\delta_{novo}}{(d,q)} \\
 & \qquad x \leftarrow x + \alpha d \\
 & \qquad r \leftarrow r - \alpha q \\
 & \qquad s \leftarrow M^{-1}r \\
 & \qquad \delta_{velho} \leftarrow \delta_{novo} \\
 & \qquad \delta_{novo} \leftarrow (r,s) \\
 & \qquad \beta \leftarrow \delta_{novo}/\delta_{velho} \\
 & \qquad d \leftarrow s + \beta d \\
 & \qquad i \leftarrow i + 1
\end{split}
\]

Apesar de, a princípio, o método convergir em \(n\) iterações (\(n\)
sendo a dimensão do problema), na prática isso pode não ocorrer devido a
instabilidade numérica - mas, mesmo deixando instabilidade numérica de
lado, para muitos problemas rodar \(n\) iterações é computacionalmente
impraticável. Além disso, em vários casos CG chega muito próximo à
solução uma quantidade \(k \ll n\) de iterações.

Por esses motivos são necessários os parâmetros \(imax\) e \(\epsilon\).

    \subsubsection{Implementação}\label{implementauxe7uxe3o}

Para fins de exemplo, veja como esse algoritmo pode ser implementado em
Julia

    \begin{tcolorbox}[breakable, size=fbox, boxrule=1pt, pad at break*=1mm,colback=cellbackground, colframe=cellborder]
\prompt{In}{incolor}{1}{\boxspacing}
\begin{Verbatim}[commandchars=\\\{\}]
\PY{k}{using} \PY{n}{LinearAlgebra} \PY{c}{\PYZsh{} para podermos usar I, a identidade, por conveniência e sem perda de eficiência computacional}
\PY{k}{function} \PY{n}{CG}\PY{p}{(}\PY{n}{A}\PY{p}{,} \PY{n}{b}\PY{p}{,} \PY{n}{x0}\PY{p}{,} \PY{n}{imax} \PY{o}{=} \PY{n}{size}\PY{p}{(}\PY{n}{A}\PY{p}{)}\PY{p}{[}\PY{l+m+mi}{1}\PY{p}{]}\PY{p}{,} \PY{n}{Minv} \PY{o}{=} \PY{n+nb}{I}\PY{p}{,} \PY{n}{ϵ} \PY{o}{=} \PY{l+m+mf}{1e\PYZhy{}5}\PY{p}{)}
    \PY{n}{i}  \PY{o}{=} \PY{l+m+mi}{0}
    \PY{n}{x}  \PY{o}{=} \PY{n}{x0}
    \PY{n}{r}  \PY{o}{=} \PY{n}{b} \PY{o}{\PYZhy{}} \PY{n}{A}\PY{o}{*}\PY{n}{x}
    \PY{n}{d}  \PY{o}{=} \PY{n}{Minv}\PY{o}{*}\PY{n}{r}
    \PY{n}{δn} \PY{o}{=} \PY{n}{r}\PY{o}{\PYZsq{}}\PY{n}{d}     
    \PY{n}{δ0} \PY{o}{=} \PY{n}{δn}
    \PY{k}{while} \PY{n}{i} \PY{o}{\PYZlt{}} \PY{n}{imax} \PY{o}{\PYZam{}\PYZam{}} \PY{n}{δn} \PY{o}{\PYZgt{}} \PY{n}{ϵ}\PY{o}{\PYZca{}}\PY{l+m+mi}{2}\PY{o}{*}\PY{n}{δ0}
        \PY{n}{q}  \PY{o}{=} \PY{n}{A}\PY{o}{*}\PY{n}{d}
        \PY{n}{α}  \PY{o}{=} \PY{n}{δn}\PY{o}{/}\PY{p}{(}\PY{n}{d}\PY{o}{\PYZsq{}}\PY{n}{q}\PY{p}{)}
        \PY{n}{x}  \PY{o}{=} \PY{n}{x} \PY{o}{+} \PY{n}{α}\PY{o}{*}\PY{n}{d}
        \PY{k}{if} \PY{p}{(}\PY{n}{i} \PY{o}{+} \PY{l+m+mi}{1}\PY{p}{)} \PY{o}{÷} \PY{l+m+mi}{50} \PY{o}{==} \PY{l+m+mi}{0}  \PY{c}{\PYZsh{} por questão de estabilidade numérica, recalculamos r a cada 50 iterações}
            \PY{n}{r} \PY{o}{=} \PY{n}{b} \PY{o}{\PYZhy{}} \PY{n}{A}\PY{o}{*}\PY{n}{x}
        \PY{k}{else}
            \PY{n}{r}  \PY{o}{=} \PY{n}{r} \PY{o}{\PYZhy{}} \PY{n}{α}\PY{o}{*}\PY{n}{q}
        \PY{k}{end}
        \PY{n}{s}  \PY{o}{=} \PY{n}{Minv}\PY{o}{*}\PY{n}{r}
        \PY{n}{δv} \PY{o}{=} \PY{n}{δn}       \PY{c}{\PYZsh{} δ velho}
        \PY{n}{δn} \PY{o}{=} \PY{n}{r}\PY{o}{\PYZsq{}}\PY{n}{s}      \PY{c}{\PYZsh{} δ novo}
        \PY{n}{β}  \PY{o}{=} \PY{n}{δn}\PY{o}{/}\PY{n}{δv}
        \PY{n}{d}  \PY{o}{=} \PY{n}{s} \PY{o}{+} \PY{n}{β}\PY{o}{*}\PY{n}{d}
        \PY{n}{i} \PY{o}{+=} \PY{l+m+mi}{1}
    \PY{k}{end}
    \PY{k}{return} \PY{n}{x}
\PY{k}{end}
\end{Verbatim}
\end{tcolorbox}

            \begin{tcolorbox}[breakable, size=fbox, boxrule=.5pt, pad at break*=1mm, opacityfill=0]
\prompt{Out}{outcolor}{1}{\boxspacing}
\begin{Verbatim}[commandchars=\\\{\}]
CG (generic function with 4 methods)
\end{Verbatim}
\end{tcolorbox}
        
    \paragraph{Teste simples}\label{teste-simples}

    \begin{tcolorbox}[breakable, size=fbox, boxrule=1pt, pad at break*=1mm,colback=cellbackground, colframe=cellborder]
\prompt{In}{incolor}{2}{\boxspacing}
\begin{Verbatim}[commandchars=\\\{\}]
\PY{n}{A}  \PY{o}{=} \PY{p}{[}\PY{l+m+mi}{3} \PY{l+m+mi}{2}\PY{p}{;} \PY{l+m+mi}{2} \PY{l+m+mi}{6}\PY{p}{]}

\PY{n}{b}  \PY{o}{=} \PY{p}{[}\PY{l+m+mi}{2}\PY{p}{;} \PY{o}{\PYZhy{}}\PY{l+m+mi}{8}\PY{p}{]}
\PY{n}{x0} \PY{o}{=} \PY{p}{[}\PY{l+m+mi}{14}\PY{p}{;} \PY{o}{\PYZhy{}}\PY{l+m+mi}{20}\PY{p}{]}

\PY{n}{x} \PY{o}{=} \PY{n}{CG}\PY{p}{(}\PY{n}{A}\PY{p}{,} \PY{n}{b}\PY{p}{,} \PY{n}{x0}\PY{p}{)}
\end{Verbatim}
\end{tcolorbox}

            \begin{tcolorbox}[breakable, size=fbox, boxrule=.5pt, pad at break*=1mm, opacityfill=0]
\prompt{Out}{outcolor}{2}{\boxspacing}
\begin{Verbatim}[commandchars=\\\{\}]
2-element Array\{Float64,1\}:
  2.0
 -2.000000000000001
\end{Verbatim}
\end{tcolorbox}
        
    Vemos que, ao menos para esse problema pequeno (para \emph{sanity
check}), o método funciona bem, apesar de haver a adição de um valor
espúrio da ordem de \(10^{-15}\):

    \begin{tcolorbox}[breakable, size=fbox, boxrule=1pt, pad at break*=1mm,colback=cellbackground, colframe=cellborder]
\prompt{In}{incolor}{3}{\boxspacing}
\begin{Verbatim}[commandchars=\\\{\}]
\PY{n}{A}\PY{o}{\PYZbs{}}\PY{n}{b}
\end{Verbatim}
\end{tcolorbox}

            \begin{tcolorbox}[breakable, size=fbox, boxrule=.5pt, pad at break*=1mm, opacityfill=0]
\prompt{Out}{outcolor}{3}{\boxspacing}
\begin{Verbatim}[commandchars=\\\{\}]
2-element Array\{Float64,1\}:
  2.0
 -2.0
\end{Verbatim}
\end{tcolorbox}
        
    \paragraph{CG não linear (NCG)}\label{cg-nuxe3o-linear-ncg}

CG pode ser usado para encontrar o mínimo de quaisquer funções contínuas
\(f\), com algumas mudanças:

\begin{enumerate}
\def\labelenumi{\arabic{enumi}.}
\tightlist
\item
  a fórmula para o resíduo muda;
\item
  é mais difícil calcular \(\alpha\) - é preciso fazer, por exemplo, uma
  busca linear;
\item
  existem diferentes escolhas possíveis para \(\beta\).
\end{enumerate}

Não entraremos em muitos detalhes aqui, mas vale fazer uma comparação
com o popular BFGS. BFGS é um método quase-Newton bem conhecido, cuja
principal característica que afeta sua performance (em particular, em
armazenamento) é a utilização de uma aproximação da Hessiana de \(f\).
Para problemas grandes, essa o armazenamento dessa aproximação pode
significar um grande custo computacional, tornando sua utilização
inviável, e o NCG uma alternativa válida.

Se a memória não for um problema, no entanto, BFGS, exceto para casos
específicos, tende a se tornar uma melhor opção (na média, uma iteração
de BFGS equivale a \(n\) de NCG em questão de convergência, de modo que
mesmo uma iteração de NCG podendo ser mais barata do que uma de BFGS, no
geral essa diferença não compensa).

A variante L-BFGS (lê-se "\emph{limited memory BFGS}") do BFGS é uma
aproximação do mesmo que apresenta menor consumo de memória e
(assintoticamente) menor tempo computacional por iteração, apesar de
convergência mais lenta.

Vale lembrar que essas são considerações gerais, no entanto, sendo que
cada problema prático pode exigir uma análise mais detalhada, apontando
qual método seria mais efetivo.

    \subsection{Exemplo de Aplicação: Resolvendo o problema de Poisson em
elementos
finitos}\label{exemplo-de-aplicauxe7uxe3o-resolvendo-o-problema-de-poisson-em-elementos-finitos}

O problema de Poisson é o de encontrar \(u\) tal que
\[
\begin{aligned}
\nabla^2 u(x) &= -f,\quad & x \in \Omega \\
u(x) &= u_D(x),\quad &x \in \partial\Omega.
\end{aligned}
\]

Onde \(u = u(x)\) é a função desconhecida, \(f = f(x)\) é uma função,
\(\nabla^2\) é o operador de Laplace, \(\Omega\) é o domínio espacial e
\(\partial\Omega\) é sua borda.

Apesar de simples, esse problema é muito importante, por exemplo, para
aplicações em física e engenharia.

    Para prosseguir pelo método de elementos finitos, precisamos reescrever
o problema em sua forma variacional. Para tanto, primeiro multiplica-se
a equação por uma função \(v\), dita \emph{função de teste}, e
integramos no domínio \[
\int_{\Omega} (\nabla^2u)v dx = \int_{\Omega} fv dx
\]

então, aplicando integração por partes e insistindo que \(v\) deve sumir
na borda, obtém-se o sistema

\[
\int_{\Omega}\nabla u \cdot \nabla v dx = \int_{\Omega} fv dx \quad \forall v \in V
\]

ou, de forma mais compacta (em que se colocam os problemas para serem
resolvidos por métodos elementos finitos no geral)

\[
a(u,v) = l(v)
\]

onde a forma bilinear \(a\) é
\(\int_{\Omega}\nabla u \cdot \nabla v dx\) e a forma linear \(v\) é
\(\int_{\Omega} fv dx\), para \(v in V\), onde \(V\) é chamado "espaço
de teste" (costuma-se insistir que \(V\) seja um espaço Sobolev
apropriado, mas por questões de brevidade vamos omitir aqui esse tipo de
detalhe técnico).

A solução \(u\) da EDP subjacente deve estar em um espaço de funções com
derivadas contínuas, mas o espaço de Sobolev posto pela formulação
variacional permite variadas descontínuas, o que tem grandes
consequências práticas, como permitir a construção de uma solução a
partir da "colagem" de funções polinomiais por partes.

    Para resolver o problema, introduzimos uma discretização e buscaremos
uma aproximação poligonal \(u_N\) para \(u\). O espaço \(\Omega\) é
discretizado em uma malha conforme (assumiremos aqui a discretização em
um complexo simplicial, mas outras discretizações são possíveis). Assim,
fazemos
\(V_N = \{v \in C^0 : v|_{s_i} \text{é linear e } v(\partial \Omega) = 0 \}\),
sendo \(s_i\) os simplexos da discretização. \(V_N\) tem dimensão
finita, sendo \(v \in V_N\) unicamente determinado por seus valores nos
pontos da malha.

    Dada uma base \(\{\phi_i\}\) para \(V_N\) (por exemplo, a base de
Lagrange usual), temos \(u_N = \sum_1^N U_j\phi_j\). Como todo
\(v \in V_N\) é combinação linear dos \(\{\phi_j\}\), vemos que a
formulação variacional é equivalente a

    \[
\int_{\Omega} u_N' \phi_k' dx = \int_{\Omega} f \phi_k dx \text{ para } k = 1,\ldots, N
\]

do que segue que \[
\sum_1^N U_j\int_{\Omega} \phi_j'\phi_k' dx = \int_{\Omega} f \phi_k dx \text{ para } k = 1,\ldots, N
\]

Assim, o problema se torn encontrar \(U \in \mathbb R^N\) resolvendo o
sistema \[
AU = F
\]

onde \(a_{ij} = \int_{\Omega} \phi_j' \phi_i' dx\) e
\(F_k = \int_{\Omega} f \phi_k dx\).

Tipicamente, a matriz \(A\) é altamente esparsa quando se escolhe uma
base apropriada (geralmente polinômios de Lagrange) e, quando se busca
solução para uma malha de refinamento alto, o problema pode atingir
dimensões muito altas (e especialmente para malhas em \(\mathbb R^3\)).
Assim, para problemas grandes o suficiente (que são na verdade problemas
de tamanho modesto para malhas em \(\mathbb R^3\)), é necessário o uso
de métodos iterativos, entre os quais CG é entre os mais populares. CG é
especialmente adequado para o problema de Poisson, que resulta em uma
matriz simétrica e positiva-definida, e pacotes para métodos de
elementos finitos populares, como o
\href{https://fenicsproject.org/}{fenics} oferecem a opção de usá-lo
como método iterativo (com precondicionador apropriado).

    \subsection{Experimentos
computacionais}\label{experimentos-computacionais}

A seguir são feitos experimentos computacionais quanto à adequação de CG
para a resolução de sistemas esparsos, com ou sem condicionador.

Para efeitos dos experimentos, foi utilizada uma matriz de Wathen. Uma
matriz de Wathen(Nx,Ny) é uma matriz de elementos finitos aleatória N
por N (fazendo \(N = 3NxNy + 2Nx + 2Ny + 1\)), sendo a "matriz de
consistência de massa" para um grade regular Nx por Ny de 8 elementos
nodais em 2 dimensões espaciais. A matriz é simétrica positiva definida
para qualquer valor (positivo) da densidade, que é escolhida
aleatoriamente.

    \begin{tcolorbox}[breakable, size=fbox, boxrule=1pt, pad at break*=1mm,colback=cellbackground, colframe=cellborder]
\prompt{In}{incolor}{4}{\boxspacing}
\begin{Verbatim}[commandchars=\\\{\}]
\PY{k}{using} \PY{n}{BenchmarkTools}\PY{p}{,} \PY{n}{MatrixDepot}\PY{p}{,} \PY{n}{IterativeSolvers}\PY{p}{,} \PY{n}{LinearAlgebra}\PY{p}{,} \PY{n}{SparseArrays}

\PY{c}{\PYZsh{} Matriz de Wathen de dimensões 30401 x 30401}
\PY{n}{A} \PY{o}{=} \PY{n}{matrixdepot}\PY{p}{(}\PY{l+s}{\PYZdq{}}\PY{l+s}{w}\PY{l+s}{a}\PY{l+s}{t}\PY{l+s}{h}\PY{l+s}{e}\PY{l+s}{n}\PY{l+s}{\PYZdq{}}\PY{p}{,} \PY{l+m+mi}{100}\PY{p}{)}
\end{Verbatim}
\end{tcolorbox}

    \begin{Verbatim}[commandchars=\\\{\}]
include group.jl for user defined matrix generators
verify download of index files{\ldots}
reading database
adding metadata{\ldots}
adding svd data{\ldots}
writing database
used remote sites are sparse.tamu.edu with MAT index and math.nist.gov with HTML
index
    \end{Verbatim}

            \begin{tcolorbox}[breakable, size=fbox, boxrule=.5pt, pad at break*=1mm, opacityfill=0]
\prompt{Out}{outcolor}{4}{\boxspacing}
\begin{Verbatim}[commandchars=\\\{\}]
30401×30401 SparseMatrixCSC\{Float64,Int64\} with 471601 stored entries:
  [1    ,     1]  =  6.2388
  [2    ,     1]  =  -6.2388
  [3    ,     1]  =  2.0796
  [202  ,     1]  =  -6.2388
  [203  ,     1]  =  -8.31839
  [303  ,     1]  =  2.0796
  [304  ,     1]  =  -8.31839
  [305  ,     1]  =  3.1194
  [1    ,     2]  =  -6.2388
  [2    ,     2]  =  33.2736
  [3    ,     2]  =  -6.2388
  [202  ,     2]  =  20.796
  ⋮
  [30199, 30400]  =  21.3736
  [30200, 30400]  =  21.3736
  [30399, 30400]  =  -6.41209
  [30400, 30400]  =  34.1978
  [30401, 30400]  =  -6.41209
  [30097, 30401]  =  3.20604
  [30098, 30401]  =  -8.54945
  [30099, 30401]  =  2.13736
  [30199, 30401]  =  -8.54945
  [30200, 30401]  =  -6.41209
  [30399, 30401]  =  2.13736
  [30400, 30401]  =  -6.41209
  [30401, 30401]  =  6.41209
\end{Verbatim}
\end{tcolorbox}
        
%     \begin{tcolorbox}[breakable, size=fbox, boxrule=1pt, pad at break*=1mm,colback=cellbackground, colframe=cellborder]
% \prompt{In}{incolor}{5}{\boxspacing}
% \begin{Verbatim}[commandchars=\\\{\}]
% \PY{k}{using} \PY{n}{UnicodePlots}
% \PY{n}{spy}\PY{p}{(}\PY{n}{A}\PY{p}{)}
% % \end{Verbatim}
% \end{tcolorbox}

%             \begin{tcolorbox}[breakable, size=fbox, boxrule=.5pt, pad at break*=1mm, opacityfill=0]
% \prompt{Out}{outcolor}{5}{\boxspacing}
% \begin{Verbatim}[commandchars=\\\{\}]
% \textbf{                      Sparsity Pattern}
% \textcolor{ansi-black-intense}{         ┌──────────────────────────────────────────┐}
%        \textcolor{ansi-black-intense}{1}\textcolor{ansi-black-intense}{ │}\textcolor{ansi-magenta}{⠻}\textcolor{ansi-magenta}{⣦}\textcolor{ansi-magenta}{⡀}⠀⠀⠀[
% 0m⠀⠀⠀⠀⠀⠀⠀⠀⠀⠀⠀⠀⠀⠀⠀⠀[
% 0m⠀⠀⠀⠀⠀⠀⠀⠀⠀⠀⠀⠀⠀⠀⠀⠀[
% 0m⠀⠀⠀⠀\textcolor{ansi-black-intense}{│} \textcolor{ansi-red}{> 0}
%         \textcolor{ansi-black-intense}{ │}⠀\textcolor{ansi-magenta}{⠈}\textcolor{ansi-magenta}{⠻}\textcolor{ansi-magenta}{⣦}\textcolor{ansi-magenta}{⡀}⠀⠀
% [0m⠀⠀⠀⠀⠀⠀⠀⠀⠀⠀⠀⠀⠀⠀⠀⠀
% [0m⠀⠀⠀⠀⠀⠀⠀⠀⠀⠀⠀⠀⠀⠀⠀⠀
% [0m⠀⠀⠀\textcolor{ansi-black-intense}{│} \textcolor{ansi-blue}{< 0}
%         \textcolor{ansi-black-intense}{ │}⠀⠀⠀\textcolor{ansi-magenta}{⠈}\textcolor{ansi-magenta}{⠻}\textcolor{ansi-magenta}{⣦}\textcolor{ansi-magenta}{⡀}
% [0m⠀⠀⠀⠀⠀⠀⠀⠀⠀⠀⠀⠀⠀⠀⠀⠀
% [0m⠀⠀⠀⠀⠀⠀⠀⠀⠀⠀⠀⠀⠀⠀⠀⠀
% [0m⠀⠀⠀\textcolor{ansi-black-intense}{│}
%         \textcolor{ansi-black-intense}{ │}⠀⠀⠀⠀⠀\textcolor{ansi-magenta}{⠈}\textcolor{ansi-magenta}{⠻}\textcolor{ansi-magenta}{⣦}[
% 35m⡀⠀⠀⠀⠀⠀⠀⠀⠀⠀⠀⠀⠀⠀⠀
% [0m⠀⠀⠀⠀⠀⠀⠀⠀⠀⠀⠀⠀⠀⠀⠀⠀
% [0m⠀⠀⠀\textcolor{ansi-black-intense}{│}
%         \textcolor{ansi-black-intense}{ │}⠀⠀⠀⠀⠀⠀⠀\textcolor{ansi-magenta}{⠈}\textcolor{ansi-magenta}{⠻}[3
% 5m⣦\textcolor{ansi-magenta}{⡀}⠀⠀⠀⠀⠀⠀⠀⠀⠀⠀⠀⠀
% [0m⠀⠀⠀⠀⠀⠀⠀⠀⠀⠀⠀⠀⠀⠀⠀⠀
% [0m⠀⠀⠀\textcolor{ansi-black-intense}{│}
%         \textcolor{ansi-black-intense}{ │}⠀⠀⠀⠀⠀⠀⠀⠀⠀\textcolor{ansi-magenta}{⠈}[35
% m⠻\textcolor{ansi-magenta}{⣦}\textcolor{ansi-magenta}{⡀}⠀⠀⠀⠀⠀⠀⠀⠀⠀⠀
% [0m⠀⠀⠀⠀⠀⠀⠀⠀⠀⠀⠀⠀⠀⠀⠀⠀
% [0m⠀⠀⠀\textcolor{ansi-black-intense}{│}
%         \textcolor{ansi-black-intense}{ │}⠀⠀⠀⠀⠀⠀⠀⠀⠀⠀⠀\textcolor{ansi-magenta}{
% ⠈}\textcolor{ansi-magenta}{⠻}\textcolor{ansi-magenta}{⣦}\textcolor{ansi-magenta}{⡀}⠀⠀⠀⠀⠀⠀⠀⠀
% [0m⠀⠀⠀⠀⠀⠀⠀⠀⠀⠀⠀⠀⠀⠀⠀⠀
% [0m⠀⠀⠀\textcolor{ansi-black-intense}{│}
%         \textcolor{ansi-black-intense}{ │}⠀⠀⠀⠀⠀⠀⠀⠀⠀⠀⠀⠀
% ⠀\textcolor{ansi-magenta}{⠈}\textcolor{ansi-magenta}{⠻}\textcolor{ansi-magenta}{⣦}\textcolor{ansi-magenta}{⡀}⠀⠀⠀⠀⠀⠀
% [0m⠀⠀⠀⠀⠀⠀⠀⠀⠀⠀⠀⠀⠀⠀⠀⠀
% [0m⠀⠀⠀\textcolor{ansi-black-intense}{│}
%         \textcolor{ansi-black-intense}{ │}⠀⠀⠀⠀⠀⠀⠀⠀⠀⠀⠀⠀
% ⠀⠀⠀\textcolor{ansi-magenta}{⠈}\textcolor{ansi-magenta}{⠻}\textcolor{ansi-magenta}{⣦}\textcolor{ansi-magenta}{⡀}⠀⠀⠀⠀
% [0m⠀⠀⠀⠀⠀⠀⠀⠀⠀⠀⠀⠀⠀⠀⠀⠀
% [0m⠀⠀⠀\textcolor{ansi-black-intense}{│}
%         \textcolor{ansi-black-intense}{ │}⠀⠀⠀⠀⠀⠀⠀⠀⠀⠀⠀⠀
% ⠀⠀⠀⠀⠀\textcolor{ansi-magenta}{⠈}\textcolor{ansi-magenta}{⠻}\textcolor{ansi-magenta}{⣦}\textcolor{ansi-magenta}{⡀}⠀⠀
% [0m⠀⠀⠀⠀⠀⠀⠀⠀⠀⠀⠀⠀⠀⠀⠀⠀
% [0m⠀⠀⠀\textcolor{ansi-black-intense}{│}
%         \textcolor{ansi-black-intense}{ │}⠀⠀⠀⠀⠀⠀⠀⠀⠀⠀⠀⠀
% ⠀⠀⠀⠀⠀⠀⠀\textcolor{ansi-magenta}{⠈}\textcolor{ansi-magenta}{⠻}\textcolor{ansi-magenta}{⣦}\textcolor{ansi-magenta}{⡀}
% [0m⠀⠀⠀⠀⠀⠀⠀⠀⠀⠀⠀⠀⠀⠀⠀⠀
% [0m⠀⠀⠀\textcolor{ansi-black-intense}{│}
%         \textcolor{ansi-black-intense}{ │}⠀⠀⠀⠀⠀⠀⠀⠀⠀⠀⠀⠀
% ⠀⠀⠀⠀⠀⠀⠀⠀⠀\textcolor{ansi-magenta}{⠈}\textcolor{ansi-magenta}{⠻}\textcolor{ansi-magenta}{⣦}[
% 35m⡀⠀⠀⠀⠀⠀⠀⠀⠀⠀⠀⠀⠀⠀⠀
% [0m⠀⠀⠀\textcolor{ansi-black-intense}{│}
%         \textcolor{ansi-black-intense}{ │}⠀⠀⠀⠀⠀⠀⠀⠀⠀⠀⠀⠀
% ⠀⠀⠀⠀⠀⠀⠀⠀⠀⠀⠀\textcolor{ansi-magenta}{⠈}\textcolor{ansi-magenta}{⠻}[3
% 5m⣦\textcolor{ansi-magenta}{⡀}⠀⠀⠀⠀⠀⠀⠀⠀⠀⠀⠀⠀
% [0m⠀⠀⠀\textcolor{ansi-black-intense}{│}
%         \textcolor{ansi-black-intense}{ │}⠀⠀⠀⠀⠀⠀⠀⠀⠀⠀⠀⠀
% ⠀⠀⠀⠀⠀⠀⠀⠀⠀⠀⠀⠀⠀\textcolor{ansi-magenta}{⠈}[35
% m⠻\textcolor{ansi-magenta}{⣦}\textcolor{ansi-magenta}{⡀}⠀⠀⠀⠀⠀⠀⠀⠀⠀⠀
% [0m⠀⠀⠀\textcolor{ansi-black-intense}{│}
%         \textcolor{ansi-black-intense}{ │}⠀⠀⠀⠀⠀⠀⠀⠀⠀⠀⠀⠀
% ⠀⠀⠀⠀⠀⠀⠀⠀⠀⠀⠀⠀⠀⠀⠀\textcolor{ansi-magenta}{
% ⠈}\textcolor{ansi-magenta}{⠻}\textcolor{ansi-magenta}{⣦}\textcolor{ansi-magenta}{⡀}⠀⠀⠀⠀⠀⠀⠀⠀
% [0m⠀⠀⠀\textcolor{ansi-black-intense}{│}
%         \textcolor{ansi-black-intense}{ │}⠀⠀⠀⠀⠀⠀⠀⠀⠀⠀⠀⠀
% ⠀⠀⠀⠀⠀⠀⠀⠀⠀⠀⠀⠀⠀⠀⠀⠀
% ⠀\textcolor{ansi-magenta}{⠈}\textcolor{ansi-magenta}{⠻}\textcolor{ansi-magenta}{⣦}\textcolor{ansi-magenta}{⡀}⠀⠀⠀⠀⠀⠀
% [0m⠀⠀⠀\textcolor{ansi-black-intense}{│}
%         \textcolor{ansi-black-intense}{ │}⠀⠀⠀⠀⠀⠀⠀⠀⠀⠀⠀⠀
% ⠀⠀⠀⠀⠀⠀⠀⠀⠀⠀⠀⠀⠀⠀⠀⠀
% ⠀⠀⠀\textcolor{ansi-magenta}{⠈}\textcolor{ansi-magenta}{⠻}\textcolor{ansi-magenta}{⣦}\textcolor{ansi-magenta}{⡀}⠀⠀⠀⠀
% [0m⠀⠀⠀\textcolor{ansi-black-intense}{│}
%         \textcolor{ansi-black-intense}{ │}⠀⠀⠀⠀⠀⠀⠀⠀⠀⠀⠀⠀
% ⠀⠀⠀⠀⠀⠀⠀⠀⠀⠀⠀⠀⠀⠀⠀⠀
% ⠀⠀⠀⠀⠀\textcolor{ansi-magenta}{⠈}\textcolor{ansi-magenta}{⠻}\textcolor{ansi-magenta}{⣦}\textcolor{ansi-magenta}{⡀}⠀⠀
% [0m⠀⠀⠀\textcolor{ansi-black-intense}{│}
%         \textcolor{ansi-black-intense}{ │}⠀⠀⠀⠀⠀⠀⠀⠀⠀⠀⠀⠀
% ⠀⠀⠀⠀⠀⠀⠀⠀⠀⠀⠀⠀⠀⠀⠀⠀
% ⠀⠀⠀⠀⠀⠀⠀\textcolor{ansi-magenta}{⠈}\textcolor{ansi-magenta}{⠻}\textcolor{ansi-magenta}{⣦}\textcolor{ansi-magenta}{⡀}
% [0m⠀⠀⠀\textcolor{ansi-black-intense}{│}
%         \textcolor{ansi-black-intense}{ │}⠀⠀⠀⠀⠀⠀⠀⠀⠀⠀⠀⠀
% ⠀⠀⠀⠀⠀⠀⠀⠀⠀⠀⠀⠀⠀⠀⠀⠀
% ⠀⠀⠀⠀⠀⠀⠀⠀⠀\textcolor{ansi-magenta}{⠈}\textcolor{ansi-magenta}{⠻}\textcolor{ansi-magenta}{⣦}[
% 35m⡀⠀\textcolor{ansi-black-intense}{│}
%    \textcolor{ansi-black-intense}{30401}\textcolor{ansi-black-intense}{ │}⠀⠀⠀⠀⠀⠀⠀⠀⠀⠀
% ⠀⠀⠀⠀⠀⠀⠀⠀⠀⠀⠀⠀⠀⠀⠀⠀
% ⠀⠀⠀⠀⠀⠀⠀⠀⠀⠀⠀⠀⠀\textcolor{ansi-magenta}{⠈}[35
% m⠻\textcolor{ansi-magenta}{⣦}\textcolor{ansi-black-intense}{│}
% \textcolor{ansi-black-intense}{         └──────────────────────────────────────────┘}
% \textcolor{ansi-black-intense}{         1}\textcolor{ansi-black-intense}{                    }\textcolor{ansi-black-intense}{
% 30401}
%                          nz = 471601
% \end{Verbatim}
% \end{tcolorbox}
        
    \begin{tcolorbox}[breakable, size=fbox, boxrule=1pt, pad at break*=1mm,colback=cellbackground, colframe=cellborder]
\prompt{In}{incolor}{6}{\boxspacing}
\begin{Verbatim}[commandchars=\\\{\}]
\PY{c}{\PYZsh{} Nível de esparsidade}
\PY{n}{count}\PY{p}{(}\PY{o}{!}\PY{n}{iszero}\PY{p}{,} \PY{n}{A}\PY{p}{)} \PY{o}{/} \PY{n}{length}\PY{p}{(}\PY{n}{A}\PY{p}{)}
\end{Verbatim}
\end{tcolorbox}

            \begin{tcolorbox}[breakable, size=fbox, boxrule=.5pt, pad at break*=1mm, opacityfill=0]
\prompt{Out}{outcolor}{6}{\boxspacing}
\begin{Verbatim}[commandchars=\\\{\}]
0.0005102687577359558
\end{Verbatim}
\end{tcolorbox}
        
    \begin{tcolorbox}[breakable, size=fbox, boxrule=1pt, pad at break*=1mm,colback=cellbackground, colframe=cellborder]
\prompt{In}{incolor}{7}{\boxspacing}
\begin{Verbatim}[commandchars=\\\{\}]
\PY{n}{b} \PY{o}{=} \PY{n}{ones}\PY{p}{(}\PY{n}{size}\PY{p}{(}\PY{n}{A}\PY{p}{,} \PY{l+m+mi}{1}\PY{p}{)}\PY{p}{)}
\PY{c}{\PYZsh{} Resolve Ax=b by CG}
\PY{n}{xcg} \PY{o}{=} \PY{n}{cg}\PY{p}{(}\PY{n}{A}\PY{p}{,} \PY{n}{b}\PY{p}{)}\PY{p}{;}
\PY{n+nd}{@benchmark} \PY{n}{cg}\PY{p}{(}\PY{o}{\PYZdl{}}\PY{n}{A}\PY{p}{,} \PY{o}{\PYZdl{}}\PY{n}{b}\PY{p}{)}
\end{Verbatim}
\end{tcolorbox}

            \begin{tcolorbox}[breakable, size=fbox, boxrule=.5pt, pad at break*=1mm, opacityfill=0]
\prompt{Out}{outcolor}{7}{\boxspacing}
\begin{Verbatim}[commandchars=\\\{\}]
BenchmarkTools.Trial:
  memory estimate:  951.36 KiB
  allocs estimate:  16
  --------------
  minimum time:     280.710 ms (0.00\% GC)
  median time:      430.303 ms (0.00\% GC)
  mean time:        438.680 ms (0.00\% GC)
  maximum time:     765.398 ms (0.00\% GC)
  --------------
  samples:          12
  evals/sample:     1
\end{Verbatim}
\end{tcolorbox}
        
    \subsubsection{Usando precondicionador de
Cholesky}\label{usando-precondicionador-de-cholesky}

    \begin{tcolorbox}[breakable, size=fbox, boxrule=1pt, pad at break*=1mm,colback=cellbackground, colframe=cellborder]
\prompt{In}{incolor}{8}{\boxspacing}
\begin{Verbatim}[commandchars=\\\{\}]
\PY{k}{using} \PY{n}{Preconditioners}
\PY{n+nd}{@time} \PY{n}{p} \PY{o}{=} \PY{n}{CholeskyPreconditioner}\PY{p}{(}\PY{n}{A}\PY{p}{,} \PY{l+m+mi}{2}\PY{p}{)}
\end{Verbatim}
\end{tcolorbox}

    \begin{Verbatim}[commandchars=\\\{\}]
  4.076178 seconds (2.43 M allocations: 151.649 MiB, 1.26\% gc time)
    \end{Verbatim}

            \begin{tcolorbox}[breakable, size=fbox, boxrule=.5pt, pad at break*=1mm, opacityfill=0]
\prompt{Out}{outcolor}{8}{\boxspacing}
\begin{Verbatim}[commandchars=\\\{\}]
CholeskyPreconditioner\{Float64,SparseMatrixCSC\{Float64,Int64\}\}([7.77401676578252
85 0.0 … 0.0 0.0; 0.0 11.520622371156024 … 0.0 0.0; … ; 0.0 0.0 …
3.048358439214821 0.0; 0.0 0.0 … 0.0 5.7811090761367625], 2)
\end{Verbatim}
\end{tcolorbox}
        
    Resolve Ax=b com precondicionador

    \begin{tcolorbox}[breakable, size=fbox, boxrule=1pt, pad at break*=1mm,colback=cellbackground, colframe=cellborder]
\prompt{In}{incolor}{9}{\boxspacing}
\begin{Verbatim}[commandchars=\\\{\}]
\PY{n}{xpcg} \PY{o}{=} \PY{n}{cg}\PY{p}{(}\PY{n}{A}\PY{p}{,} \PY{n}{b}\PY{p}{,} \PY{n}{Pl}\PY{o}{=}\PY{n}{p}\PY{p}{)}
\PY{c}{\PYZsh{} same answer?}
\PY{n}{norm}\PY{p}{(}\PY{n}{xcg} \PY{o}{\PYZhy{}} \PY{n}{xpcg}\PY{p}{)}
\end{Verbatim}
\end{tcolorbox}

            \begin{tcolorbox}[breakable, size=fbox, boxrule=.5pt, pad at break*=1mm, opacityfill=0]
\prompt{Out}{outcolor}{9}{\boxspacing}
\begin{Verbatim}[commandchars=\\\{\}]
5.306315159734449e-7
\end{Verbatim}
\end{tcolorbox}
        
    \subparagraph{CG foi, neste exemplo \textgreater{} vezes 10 mais lento
do que CG}\label{cg-foi-neste-exemplo-vezes-10-mais-lento-do-que-cg}

O que é curioso, porque parece que em outros computadores o resultado é
o inverso (CG com precondicionador (PCG) é \textgreater{} 10 vezes mais
rápido).

    \begin{tcolorbox}[breakable, size=fbox, boxrule=1pt, pad at break*=1mm,colback=cellbackground, colframe=cellborder]
\prompt{In}{incolor}{10}{\boxspacing}
\begin{Verbatim}[commandchars=\\\{\}]
\PY{n+nd}{@benchmark} \PY{n}{cg}\PY{p}{(}\PY{o}{\PYZdl{}}\PY{n}{A}\PY{p}{,} \PY{o}{\PYZdl{}}\PY{n}{b}\PY{p}{,} \PY{n}{Pl}\PY{o}{=}\PY{o}{\PYZdl{}}\PY{n}{p}\PY{p}{)}
\end{Verbatim}
\end{tcolorbox}

            \begin{tcolorbox}[breakable, size=fbox, boxrule=.5pt, pad at break*=1mm, opacityfill=0]
\prompt{Out}{outcolor}{10}{\boxspacing}
\begin{Verbatim}[commandchars=\\\{\}]
BenchmarkTools.Trial:
  memory estimate:  951.36 KiB
  allocs estimate:  16
  --------------
  minimum time:     4.958 s (0.00\% GC)
  median time:      5.235 s (0.00\% GC)
  mean time:        5.235 s (0.00\% GC)
  maximum time:     5.513 s (0.00\% GC)
  --------------
  samples:          2
  evals/sample:     1
\end{Verbatim}
\end{tcolorbox}
        
    \subsection{Referências}\label{referuxeancias}

Para escrever este documento, foram usadas, primariamente, as seguintes
referências:

\begin{itemize}
\tightlist
\item
  An Introduction to the Conjugate Gradient Method Without the Agonizing
  Pain, de Jonathan Richard Shewchuk, primariamente para o
  desenvolvimento teórico
  \href{http://www.cs.cmu.edu/~quake-papers/painless-conjugate-gradient.pdf}{disponível
  aqui}
\item
  Notas de aula de Hua-Zhou, primariamente para a implementação
  computacional
  \href{http://hua-zhou.github.io/teaching/biostatm280-2019spring/slides/16-cg/cg.html}{disponível
  aqui}
\item
  Documentação da função Wathen para a sua utilização
  \href{http://www.netlib.org/templates/mltemplates.v1_1/wathen.m}{disponível
  aqui} (foi utilizada a documentação da função mas não a função
  indicada, que foi feita para MatLab)
\item
  Galerkin Approximations and Finite Element Methods, de Ricardo G.
  Durán, para alguns detalhes quanto a métodos de elementos finitos
  \href{http://mate.dm.uba.ar/~rduran/class_notes/fem.pdf}{disponível
  aqui}
\item
  Solving PDEs in Python - The FEniCS Tutorial Volume 1, de Hans Peter
  Langtangen e Anders Logg, para outros detalhes quanto a métodos de
  elementos finitos para o problema de Poisson,
  \href{https://fenicsproject.org/pub/tutorial/pdf/fenics-tutorial-vol1.pdf}{disponível
  aqui}
\end{itemize}


    % Add a bibliography block to the postdoc
    
    
    
\end{document}
